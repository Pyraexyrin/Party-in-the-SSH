\chapter[Termina]{Shell$\,\to\,$Termina}
Pour le plaisir, on se fait une commande Easter-Egg. Notre programme final va s'appeler Termina (sans le "l", c'est volontaire). Termina est le nom de la région du jeu "The Legend of Zelda : Majora's Mask". La commande (interne) permettant de lancer cet Easter-Egg est donc... "majora". Allez au prochain paragraphe si vous ne voulez pas être spoilés. La commande affiche un compteur (qui augmente) comptant le nombre de jours et d'heures passées depuis la date de rendu, le lundi 11 janvier, à minuit. Bon, j'avoue n'avoir implémenté qu'un calcul simplifié... Dès qu'on changera de mois, ça ne fonctionnera plus. Mais je vais pas batailler pour ça. Ça n'a pas pour objectif d'être pérenne.
\\\\
Plus sérieusement, je repasse sur le code afin de commenter un peu mieux, et il faut que je modifie le Makefile (changer le nom du programme) et le Readme (indiquer comment exécuter le programme). Une fois fait, je fais un dernier commit avant d'attaquer le gros morceau : "remote". J'ai d'ailleurs enfin utilisé le fichier Evaluation.c. J'ai laissé tout le code fourni de base dans Shell.c, et déplacé tous les ajouts dans Evaluation.c.
\\\\
\textbf{> You've met with a terrible fate, haven't you ?}