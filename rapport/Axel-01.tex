\chapter[Initialisation]{Initialisation du projet}
\textbf{> Initialisation de l'ornithorynque}\footnote{Il s'agit ici du nom d'un commit. Ils sont indiqués, pour que vous puissiez vous repérer dans le temps.}
\\\\
Dépôt pour la PRS sur mon GitHub. C'est "Party in the SSH". Ya tous les fichiers de base. Notre mission pour l'instant ? Il faut réfléchir à comment faire ce Remote Shell.
Normalement, presque tout le reste sera bon, puisque je l'avais faite l'année dernière. Je ne l'ai pas encore fait, je réfléchis au Remote Shell cette après-midi. Éventuellement, je m'occupe du mini-shell ce soir.
\\\\
Ce que j'entends par "réfléchir au Remote Shell", c'est réfléchir à, techniquement, comment mettre en place tout ça. Pas besoin de code, des dessins suffisent (et pourquoi pas du pseudo-code).
\\\\
Voilà, n'oubliez pas qu'il y a un rapport final à faire, donc prenez des notes sur absolument tout ! Le rapport final sera découpé en chapitres. Exemple :
\\\\
1/ <titre1> (<auteur1>)
\\2/ <titre2> (<auteur2>)
\\...
\\\\
Ainsi de suite. C'est faisable facilement en LaTeX, avec des import ou un truc du genre. M'en souviens plus. Lilian tu me confirmeras. On aura le fichier Rapport.tex à compiler à la racine, puis un dossier chapters dans lequel on mettra les chapitres, qu'on nommera "<auteur>-01", "<auteur>-02", et la numérotation est personnelle (chacun aura son 01, 02... on triera à la fin l'ordre final). Avantage de faire comme ça ? Tout les trucs compliqués de LaTeX seront dans le compilable, et tout le reste sera plus simple. Faut pas hésiter à mettre des images, dessins, plein de bordel pour montrer qu'on gère. Par contre, c'est 15 pages max. Donc on se donne 5 pages A4 chacun environ (juste un ordre d'idée, ça dépendra ensuite).
\\\\
Et n'oubliez pas, il y a une grille d'auto-évaluation à la fin du Sujet. On a un aperçu de la difficulté de chacune des parties, et surtout, de la proportion de chacun dans la note finale. Même si le Remote Shell (par exemple) n'est pas fonctionnel, il ne sera pas si pénalisant.